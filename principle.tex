\documentclass[11pt]{article}
\usepackage[utf8]{inputenc}
\usepackage[a4paper, total={6in, 10in}]{geometry}

\pagenumbering{arabic} 

\title{\textbf{PIC Numeric Oscilloscope}}
\author{Matthieu 'Freeman' Vienot}
\date{}
\begin{document}

\maketitle

\section{What is this ?}
\subsection {\Large\bfseries{Short presentation}}
Do you practice electronics ? If you're reading this, it is very much likely. Have you ever been in need of an oscilloscope ? Have you ever been disappointer by the price of oscilloscopes online ? Well, no more.

This project (accessible at \emph{http://github.com/iLambda/oscillo-pic18}) is a numeric oscilloscope made with PIC18F4550.

\subsection{Feedback}
I hope you will enjoy reading this document as much as I enjoyed creating it. If you have comments, suggestions or wish to report an issue you are experiencing - contact us at: \emph{me@ilambda.io}.

\section {How does it work ?}

This section details the inner workings of the oscilloscope.

\subsection{Input}

To draw the function provided to the oscilloscipe, it first needs to be processed. There is two stages of input : the modulation stage, where the signal gets scaled and clamped, and the numerical acquisition, where the signal gets transformed to binary data.

\subsubsection{Modulation stage}

This stage modulates the input signal. The input voltage can range from -15V to 15V. The output voltage of this stage needs to go from 0V (LOW) to 5V (HIGH).

A series of operational amplifiers are wired together to clamp the input tension to $[-15V;15V]$. Then, the signal is multiplicated to a scale constant.

The output tension $U_{out}(t)$ can be expressed in function of the input voltage $U_{in}(t)$ :

\[ U = clamp((U_{in} - U_{zero}s) \cdot \frac{U_{scale}}{10}, [-15;15]) \]



\subsubsection{Numerical acquisition}

\subsection{Output}

\subsubsection{Composite protocol}

\subsubsection{Graphical interface}


\section {How do I make one ?}

\subsection{Making the circuit}

\subsection{Program the PIC}

\subsection{Add the components}



\end{document}

