\documentclass[11pt]{article}
\usepackage[utf8]{inputenc}
\usepackage[a4paper, total={6in, 10in}]{geometry}

\pagenumbering{arabic}

\title{\textbf{PIC Numeric Oscilloscope}}
\author{Matthieu 'Freeman' Vienot}
\date{}
\begin{document}

\maketitle

\section{Qu'est-ce que ceci ?}
\subsection {\Large\bfseries{Courte présentation}}
Pratiquez vous l'electronique ? Si tel est le cas, avez vous déja eu le besoin de recourir a un oscilloscope pour un de vos projets ? Avez vous déja été repoussé par le prix élevé des oscilloscopes du commerce ? Si oui, ne cherchez plus.

Ce projet (accessible at \emph{http://github.com/iLambda/oscillo-pic18}) est un oscilloscope numérique réalisé à base de PIC18F4550. Son but est de détailler la fabrication d'un petit oscilloscope sans prétention destiné aux amateurs.

\subsection{Commentaires}
Si vous avez des remarques ou des commentaires - envoyez un mail a l'adresse \emph{me@ilambda.io}.

\section {Comment ça marche?}

Cette section a pour but de détailler le fonctionnemennt interne de l'oscilloscope.

\subsection{Entrée}

Afin de dessiner a l'écran la tension fournie a l'oscilloscope, il faut qu'elle soit préalablement traitée. Deux étages sont ainsi placés avant le traitement : l'étage de modulation, où le signal est translaté et mis à l'échelle, et l'étage d'acquisition numérique, ou le signal est transformé en données numériques.

\subsubsection{Étage de modulation}

Cet étage module le signal d'entrée. Ce dernier peut être compris entre -15V et 15V. La tension de sortie de cet étage va de 0V (état LOW) à 5V (état HIGH).

Plusieurs amplificateurs opérationnels sont connectés ensemble afin de contraindre la tension d'entrée entre 15V et -15V. Ensuite, le signal est multiplié par une constante d'échelle.

La tension de sortie $U_{out}(t)$ peut être exprimée en fonction de la tension d'entrée $U_{in}(t)$ :

\[ U = clamp_{[-15;15]}((U_{in} - U_{zero}) \cdot \frac{U_{scale}}{10}) \]



\subsubsection{Acquisition numérique}

\subsection{Sortie}

\subsubsection{Protocole vidéo composite}

\subsubsection{Interface graphique}


\section {Comment réaliser ce montage ?}

\subsection{Fabriquer le circuit}

\subsection{Programmer le PIC}

\subsection{Ajouter les composants}



\end{document}
